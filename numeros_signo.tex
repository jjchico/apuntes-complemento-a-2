\documentclass[spanish,a4paper,12pt,titlepage]{article}

\usepackage[spanish]{babel}
\usepackage[utf8]{inputenc}
\usepackage{amsmath}
\usepackage{amsthm}
\usepackage{amssymb}
\usepackage{natbib}
\usepackage{chngcntr}
\usepackage{authblk}
\usepackage{hyperref}

\newtheorem{theorem}{Teorema}%[section]
\newtheorem{proposition}[theorem]{Proposición}
\newtheorem{lemma}[theorem]{Lema}
\newtheorem{corollary}[theorem]{Corolario}
\theoremstyle{definition}
\newtheorem{definition}{Definición}%[section]
\newtheorem{example}{Ejemplo}%[section]
\theoremstyle{remark}
\newtheorem{case}{Caso}

\counterwithin*{case}{theorem}

\newcommand{\bbZ}{\mathbb{Z}}

\title{Representación binaria de números enteros en complemento a 2}
\author{Jorge Juan Chico}
\affil{\textit{jjchico@dte.us.es}}
\affil{Departamento de Tecnología Electrónica. Universidad de Sevilla}
%\date{Octubre, 2016}

\begin{document}

\maketitle
\newpage
\tableofcontents
\newpage

%%%%%%%%%%%%%%%%%%%%%%%%%%%%%%%%%%%%%%%%%%%%%%%%%%%%%%%%%%%%%%%%%%%%%%%%%%%%%%
\phantomsection{}
\addcontentsline{toc}{section}{Prefacio}
\section*{Prefacio}
%%%%%%%%%%%%%%%%%%%%%%%%%%%%%%%%%%%%%%%%%%%%%%%%%%%%%%%%%%%%%%%%%%%%%%%%%%%%%%

Este documento contiene un conjunto de propiedades útiles para la representación binaria de números enteros en complemento a 2. El documento se centra en las propiedades y sus demostraciones con breves notas sobre sus aplicaciones y/o utilidad.

La intención de este documento es servir como material complementario en cursos que incluyan la representación en complemento a 2 en sus contenidos.

Este obra está bajo una licencia de Creative Commons Reconocimiento-CompartirIgual 4.0 Internacional. Usted es libre de compartir y modificar esta obra siempre que cite al autor original y comparta la obra derivada con la misma licencia.\footnote{
Vea \url{http://creativecommons.org/licenses/by-sa/4.0/} para más información.
}

Puede encontrar el código fuente de este documento en:

\url{https://github.com/jjchico/apuntes-complemento-a-2}.

%%%%%%%%%%%%%%%%%%%%%%%%%%%%%%%%%%%%%%%%%%%%%%%%%%%%%%%%%%%%%%%%%%%%%%%%%%%%%%
\section{Notación posicional}
%%%%%%%%%%%%%%%%%%%%%%%%%%%%%%%%%%%%%%%%%%%%%%%%%%%%%%%%%%%%%%%%%%%%%%%%%%%%%%

En este apartado se presentan algunas propiedades básicas de la representación posicional de números que se usarán en el resto de apartados.

\begin{definition}[Representación en notación posicional]
 Dado $x \in \bbZ, x \ge 0$, la representanción en notación posicional
 de $x$ en la base numérica $b \in \bbZ, b > 1$, queda definida por el
 conjunto de cifras $\{x_{n-1}, x_{n-2}, \ldots, x_1, x_0\}$ con $x_i \in
 \bbZ, 0 \le x_i < b$, tales que:
  \[
    x = x_{n-1} b^{n-1}+x_{n-2} b^{n-2}+ \cdots + x_1 b^1+x_0
  \]
\end{definition}

\begin{theorem}\label{theorem-maxn}
  El mayor número $x \in \bbZ, x \ge 0$, representable con $n$ cifras en la
  base $b$ es
  \[
    x = b^n - 1
  \]

  \begin{proof}
    Dado que las cifras $x_i$ que forman la representación de $x$ en la
    base $b$ son todas menores que $b$, el mayor número representable es el
    que se obtiene cuando todas las cifras tienen su valor máximo $b-1$, por
    tanto, el mayor número representable con $n$ cifras es:
    \begin{align*}
      \sum_{i=0}^{n-1} (b-1)b^i
      = \sum_{i=0}^{n-1} \left(b^{i+1}-b^i\right)
      = \sum_{i=0}^{n-1} b^{i+1} - \sum_{i=0}^{n-1} b^i\\
      = \sum_{i=1}^{n} b^{i} - \sum_{i=0}^{n-1} b^i
      = b^n + \sum_{i=1}^{n-1} b^{i} - \sum_{i=1}^{n-1} b^i - 1
      = b^n - 1
    \end{align*}
  \end{proof}
\end{theorem}

\begin{corollary}\label{corollary-pos-digit}
  Cada unidad de la cifra $i$-ésima de la notación posicional en base $b$
  tiene un peso mayor al mayor valor representable con el resto de cifras
  menos significativas.
  \begin{proof}
    El peso de cada unidad de la cifra $i$-ésima es $b^i$ y según
    el Teorema~\ref{theorem-maxn}, el mayor número representable con las
    $i-1$ cifras menos significativas que la cifra $i$ es $b^i-1$ por lo
    que queda demostrado el teorema.
  \end{proof}
\end{corollary}

\begin{theorem}\label{theorem-pos-uniq}
  Dado $x \in \bbZ, x \ge 0$, la representación posicional de $x$ en
  la base $b$ es única.
  \begin{proof}
    Supongamos que hay dos representaciones distintas de $x$, esto es:
    \[
      x = \sum_{i=0}^{n-1} x_i b^i = \sum_{i=0}^{n-1} x'_i b^i
    \]
    tal que $x_i \neq x'_i$ para algún $i, 0 \le i \le n-1$. Sea $j$ el
    mayor índice tal que $x_j \neq x'_j$, por tanto se tiene que:
    \begin{align*}
      |x_j-x'_j| \ge 0 \\
      x = x_j b^j + \sum_{i=0}^{j-1} x_i b^i
        = \sum_{i=0}^{j-1} x'_i b^i \\
      x-x = 0 = (x_j-x'_j) b^j + \sum_{i=0}^{j-1} (x_i-x'_i)b^i \\
      \left| (x_j-x'_j) b^j \right| =
        \left| \sum_{i=0}^{j-1} (x_i-x'_i)b^i \right|
    \end{align*}
    Por lo tanto, se tiene que:
    \begin{align*}
      b^j \le \left| (x_j-x'_j) b^j \right|
        = \left| \sum_{i=0}^{j-1} (x_i-x'_i)b^i \right| \le
        \sum_{i=0}^{j-1} |(x_i-x'_i)|b^i \\
        \le \sum_{i=0}^{j-1} (b-1)b^i = b^j - 1
    \end{align*}
    lo cual es una contradicción y, por tanto, no existe el mayor índice
    $j$ tal que $x_j \neq x'_j$ y no existen cifras que sean diferentes en
    dos representaciones en notación posicional de $x$.
  \end{proof}
\end{theorem}

%%%%%%%%%%%%%%%%%%%%%%%%%%%%%%%%%%%%%%%%%%%%%%%%%%%%%%%%%%%%%%%%%%%%%%%%%%%%%%
\section{Representación en complemento a 2}
%%%%%%%%%%%%%%%%%%%%%%%%%%%%%%%%%%%%%%%%%%%%%%%%%%%%%%%%%%%%%%%%%%%%%%%%%%%%%%

La representación en complemento a 2 es uno de los sistemas binarios que se emplean para representar número enteros con signo (positivos y negativos) empleando palabras binarias. Otros sistemas de representación de números con signo son la representación signo-magnitud, la representación en complemento a 1 y las representaciones en exceso.

El interés de la representación en complemento a 2 radica en que simplifica el diseño de circuitos aritméticos que realicen operaciones con números enteros con signo. Por este motivo, la representación en complemento a 2 es la forma más extendida de codificación de números enteros con signo en los computadores y en los lenguajes de programación tradicionales y actuales.

En este apartado se define la representación en complemento a 2 y se introducen sus propiedades básicas.

\begin{definition}[Representación en complemento a 2]\label{def-rep-ca2}
  Dado $x \in \bbZ$ tal que $-2^{n-1} \le x \le 2^{n-1}-1$, la representación en complemento a 2 de $x$ con $n$ cifras binarias (bits) es una palabra binaria de $n$ bits y de magnitud $RC2_n(x)$ tal que:
  \[
    RC2_n(x) =
      \begin{cases}
        x     & \text{si } x \ge 0\\
        2^n+x & \text{si } x < 0
      \end{cases}
  \]
\end{definition}

\begin{definition}\label{def-rep-ca2_rep}
  Se dice que $x$ es representable en complemento a 2 con $n$ bits si $-2^{n-1} \le x \le 2^{n-1}-1$.
\end{definition}

\begin{definition}[Complemento a 2 de números enteros]
  Dado $x \in \bbZ, 0 \le x \le 2^n-1$, se define el complemento a 2 con $n$ bits de $x$, $C2_n(x)$, como:
  \[
    C2_n(x) = 2^n - x
  \]
\end{definition}

\begin{corollary}
  La magnitud de la representación en complemento a 2 de $x$ con $n$ bits, $RC2_n(x)$, puede escribirse como:
  \[
    RC2_n(x) = \begin{cases}
      x        & \text{si } x \ge 0\\
      C2_n(-x) & \text{si } x < 0
    \end{cases}
  \]
\end{corollary}

\begin{theorem}\label{theorem-sign-bit}
  Si $x$ es representable en complemento a 2 con $n$ bits, el bit más significativo de la representación en complemento a 2 de $x$ es 0 si $x \ge 0$, y 1 si $x < 0$.
  \begin{proof}
    \begin{case}
      $x \ge 0$.
      \[RC2_n(x)= x = x_{n-1}2^{n-1}+x_{n-2}2^{n-2}+\cdots+x_0\]
      Dado que $x$ es representable en complemento a 2, por la definición~\ref{def-rep-ca2_rep} se tiene que $x<2^{n-1}$ y, por el corolario~\ref{corollary-pos-digit}, debe ser $x_{n-1}=0$.
    \end{case}
    \begin{case}
      $x<0$.
      \[
        RC2_n(x)= 2^n+x = x_{n-1}2^{n-1}+x_{n-2}2^{n-2}+\cdots+x_0
      \]
      Dado que $x$ es representable en complemento a 2, por la definición~\ref{def-rep-ca2_rep} se tiene que
      \[
        -2^{n-1} \le x < 0  \qquad 2^{n-1} \le 2^n + x < 2^n
      \]
      y, por el corolario~\ref{corollary-pos-digit}, debe ser $x_{n-1}=1$.
    \end{case}
  \end{proof}
\end{theorem}

\begin{definition}[Bit de signo en la representación en complemento a 2]
  Dado $x \in \bbZ$, se denomina \emph{bit de signo} de la representación en complemnto a 2 con $n$ bits de $x$ al bit más significativo de la representación en complemento a 2 con $n$ bits de $x$.
\end{definition}

\begin{theorem}[Cálculo del Opuesto]
  Dado $x \in \bbZ$ representable en complemento a 2 con $n$ bits, $x \ne 0$ y $x \ne -2^{n-1}$, la magnitud de la representación en complemento a 2 de $-x$ es el complemento a 2 de la magnitud de la representación en complemento a 2 de $x$, siempre que $-x$ sea representable en complemento a 2. Esto es:
  \[RC2_n(-x) = C2_n(RC2_n(x))\]
  En otras palabras, la representación en complemento a 2 de $-x$ puede obtenerse simplemente complementando a 2 la representación en complemento a 2 de $x$.

  \begin{proof}
    Si $0<x<2^{n-1}$, se tiene que $RC2_n(x)=x$ y, además:
    \[
      RC2_n(-x) = 2^n - x = C2_n(x) = C2_n(RC2_n(x))
    \]
    Por otro lado, si $-2^{n-1}<x<0$, se tiene que $RC2_n(x)=2^n+x$ y, además:
    \[
      RC2_n(-x) = -x = 2^n - 2^n - x = 2^n - (2^n+x) = C2_n(RC2_n(x))
    \]
  \end{proof}
\end{theorem}

%%%%%%%%%%%%%%%%%%%%%%%%%%%%%%%%%%%%%%%%%%%%%%%%%%%%%%%%%%%%%%%%%%%%%%%%%%%%%%
\section{Suma y desbordamiento en complemento a 2}
%%%%%%%%%%%%%%%%%%%%%%%%%%%%%%%%%%%%%%%%%%%%%%%%%%%%%%%%%%%%%%%%%%%%%%%%%%%%%%

La principal ventaja de la representación en complemento a 2 es la facilidad para sumar números usando sus representaciones en complemento a 2. El siguiente teorema recoge la propiedad fundamental de la suma en complemento a 2 y dice, básicamente, que la representación en complemento a 2 de $x+y$ se obtiene sumando directamente las representaciones en complemento a 2 de $x$ y de $y$.

\begin{theorem}[Regla del la suma]\label{theorem-sum-law}
  Dados $x, y \in \bbZ$ tales que $x$, $y$ y $x+y$ son representables en complemento a 2 con $n$ bits, se cumple que:
  \[
    RC2_n(x+y) = [RC2_n(x) + RC2_n(y)] \mod 2^n
  \]
  \begin{proof}
    Distinguimos varios casos:

    \begin{case}
      $0 \le x,y$

      Dado que $x$ y $y$ son positivos, por la definición~\ref{def-rep-ca2} se tiene que:
      \[
        [RC2_n(x)+RC2_n(y)] \mod 2^n = (x + y) \mod 2^n
      \]
      Por otra parte, ya que $x+y$ es necesariamente positivo y representable en complemento a 2, se tiene que $0 \le x + y < 2^{n-1}$ y por tanto:
      \[
        [RC2_n(x)+RC2_n(y)] \mod 2^n = x + y = RC2_n(x+y)
      \]

      \begin{case}
        $x \ge 0; y < 0; x+y \ge 0$
      \end{case}

      \[
        [RC2_n(x)+RC2_n(y)] \mod 2^n = (x + 2^n + y) \mod 2^n
      \]
      Dado que $x+y \ge 0$, se tiene que:
      \[
        [RC2_n(x)+RC2_n(y)] \mod 2^n = (x + y) = RC2_n(x+y)
      \]
    \end{case}
    \begin{case}
      $a \ge 0; b < 0; x+y < 0$

      \[
        [RC2_n(x)+RC2_n(y)] \mod 2^n = (x + 2^n + y) \mod 2^n
      \]
      Dado que $x+y < 0$, se tiene que:
      \[
        [RC2_n(x)+RC2_n(y)] \mod 2^n = 2^n + x + y = RC2_n(x + y)
      \]
    \end{case}

    \begin{case}
      $x < 0; y < 0$

      \[
        [RC2_n(x)+RC2_n(y)] \mod 2^n = (2^n + x + 2^n + y) \mod 2^n
      \]

      Dado que tanto $x$ como $y$ son representables en complemento a 2
      se tiene que $-2^{n-1} < x+y < 0$, por tanto, $-2^n < x+y < 0$
      y $0 < 2^n + x + y < 2^n$, luego:
      \[
        [RC2_n(x)+RC2_n(y)] \mod 2^n = 2^n + x + y = RC2_n(x + y)
      \]
    \end{case}
  \end{proof}
\end{theorem}

\begin{definition}[Desbordamiento en complemento a 2]
  Dados $x, y \in \bbZ$ representables en complemento a 2 con $n$ bits, se dice que la suma en complemento a 2 con $n$ bits de $x$ y $y$ produce desbordamiento si $x+y$ no es representable en complemento a 2 con $n$ bits.
\end{definition}

\begin{theorem}[Regla del desbordamiento]\label{theorem-overflow-law}
  Dados $x, y \in \bbZ$ representables en complemento a 2 con $n$ bits, la suma $x+y$ produce desbordamiento si y sólo si los bits de signo de las representaciones en complemento a 2 de $x$ y $y$ coinciden, pero difieren del bit más significativo de $s = [RC2_n(x)+RC2_n(y)] \mod 2^n$.
  \begin{proof}
    Distinguimos varios casos:

    \begin{case}
      $0 \le x,y$

      En este caso, $x$ y $y$ tienen idénticos bits de signo e iguales a $0$. Dado que $x$ y $y$ son positivos y representables en complemento a 2, se tiene que $0 \le x+y < 2^n$ y $s = (x + y)$.

      Si $x+y < 2^{n-1}$, $x+y$ es representable en complemento a 2 con $n$ bits, su bit más significativo es $0$ y coincide con los bits de signo de $x$ y $y$.

      Por otra parte, si $2^{n-1} \le x+y < 2^n$, $x+y$ no es representable en complemento a 2 con $n$ bits, su bit más significativo es $1$ y no coincide con los bits de signo de $x$ y $y$.
    \end{case}

    \begin{case}
      $x \ge 0; y < 0$

      Este caso es equivalente al caso en que $y \ge 0; x < 0$ el cual tiene una demostración equivalente.

      En este caso, $x$ y $y$ tienen distintos bits de signo. Dado que $x$ e $y$ son representables en complemento a 2 con $n$ bits, se tiene que $0 \le x < 2^{n-1}$ y $2^{n-1} \le y < 0$, por tanto:

      \[
        2^{n-1} \le x + y < 2^{n-1}
      \]

      Luego $x+y$ es representable en complemento a 2 con $n$ bits y la suma de $x$ e $y$ no produce desbordamiento.
    \end{case}

    \begin{case}
      $x < 0; y < 0$

      En este caso, $x$ e $y$ tienen idénticos bits de signo e iguales a $1$. Dado que $x$ e $y$ son negativos y representables en complemento a 2, se tiene que $s = (2^n + x + 2^n + y) \mod 2^n$ y $-2^n \le x+y < 0$. Por tanto:
      \begin{align*}
        2^n + 2^n - 2^n-1 \le 2^n + 2^n + x + y < 2^n + 2^n\\
        2^n \le 2^n + 2^n + x + y < 2^n + 2^n\\
        s = (2^n + 2^n + x + y) \mod 2^n = 2^n + x + y
      \end{align*}

      Si $-2^{n-1} \le x+y < 0$, $x+y$ es representable en complemento a 2 con $n$ bits y se tiene que $2^{n-1} \le s < 2^n$ y por tanto el bit más significativo de $s$ es $1$ y coincide con los bits de signo de $x$ e $y$.

      Por otra parte, si $-2^n \le x+y < -2^{n-1}$, $x+y$ no es representable en complemento a 2 con $n$ bits y se tiene que $0 \le s < 2^{n-1}$ y por tanto el bit más significativo de $s$ es $0$ y no coincide con los bits de signo de $x$ y $y$.
    \end{case}
  \end{proof}
\end{theorem}

%%%%%%%%%%%%%%%%%%%%%%%%%%%%%%%%%%%%%%%%%%%%%%%%%%%%%%%%%%%%%%%%%%%%%%%%%%%%%%
\section{Uso de sumadores de magnitudes con complemento a 2}
%%%%%%%%%%%%%%%%%%%%%%%%%%%%%%%%%%%%%%%%%%%%%%%%%%%%%%%%%%%%%%%%%%%%%%%%%%%%%%

Las propiedades de la suma en complemento a 2 tienen su principal aplicación a la hora de diseñar circuitos electrónicos sumadores. Gracias a las propiedades de la suma en complemento a 2, el diseño de estos circuitos se simplifica de forma notable. A continuación se presentan las definiciones y propiedades relacionadas con la suma en complemento a 2 aplicadas a los sumadores de magnitudes.

\begin{definition}[Sumador de magnitudes]
  Un sumador de magnitudes de $n$ bits es un sistema con dos entradas de $n$ bits que codifican las magnitudes $x$ e $y$ en base 2, y una saliza $z$ de $n$ bits que codifica en base 2 el resultado de sumar las magnitudes de las entradas de forma que:
  \[
    z = (x + y) \mod 2^n
  \]
\end{definition}

El siguiente corolario es una aplicación directa de la regla de la suma en complemento a 2 a sumadores de magnitudes. De esta forma, es posible usar dicho sumador, inicialmente diseñado para números positivos, para sumar número positivos y negativos sin más que usar sus representaciones en complemento a 2.

\begin{corollary}
  Un sumador de magnitudes de $n$ bits cuyos operandos son las representaciones en complemento a 2 con $n$ bits de $x, y \in \bbZ$, produce la representación en complemento a 2 con $n$ bits de $x+y$, siempre que $x+y$ sea representable en complemento a 2 con $n$ bits.
  \begin{proof}
    Si al sumador de magnitudes se le suministran en sus entradas las representaciones en complemento a 2 de $x$ e $y$, el sumador proporcionará una salida igual a:
    \[
      z = [RC2_n(x)+RC2_n(y)] \mod 2^n
    \]
    Si $x+y$ es una cantidad representable en complemento a 2, por el teorema~\ref{theorem-sum-law} se tiene que:
    \[
      z = RC2_n(x + y)
    \]
  \end{proof}
\end{corollary}

El siguiente corolario introduce un método práctico para detectar cuando una operación de suma en complemento a 2 realizada mediante un sumador de magnitudes produce desbordamiento. A partir de este resultado de puede llegar de forma sencilla a una implementación práctica de un circuito que detecte esta condición de desbordamiento.

\begin{corollary}
  Las suma de las representaciones en complemento a 2 de dos números $x, y \in \bbZ$ de $n$ bits realizada por un sumador de magnitudes de $n$ bits produce desbordamiento si y sólo si los bits de signo de $x$ e $y$ coinciden pero son diferentes del bit más significativo proporcionado por la salida del sumador.
  \begin{proof}
    Teniendo en cuenta que la salida $z$ del sumador es:
    \[
      z = [RC2_n(x)+RC2_n(y)] \mod 2^n
    \]
    este corolario es una mera expresión alternativa del teorema~\ref{theorem-overflow-law}.
  \end{proof}
\end{corollary}

%% TODO: Detección del desbordamiento mediante acarreos. Sería necesario introducir el concepto de acarreo, probablemente en un apartado breve sobre aritmética binaria.

%%%%%%%%%%%%%%%%%%%%%%%%%%%%%%%%%%%%%%%%%%%%%%%%%%%%%%%%%%%%%%%%%%%%%%%%%%%%%%
\section{Complemento a 1 y relación con el complemento a 2}
%%%%%%%%%%%%%%%%%%%%%%%%%%%%%%%%%%%%%%%%%%%%%%%%%%%%%%%%%%%%%%%%%%%%%%%%%%%%%%

El complemento a 1 de números enteros es una operación similar al complemento a 2. A partir de esta operación es posible definir una representación en complemento a 1 con características y propiedades similares a la representación en complemento a 2. En este documento se introduce la la operación del complemento a 1 sólo como una herramienta práctica para la obtención de métodos sencillos para obtener el complemento a 2 de un número entero de forma directa a partir de su representación en base 2.

\begin{definition}[Complemento a 1 de números enteros]\label{def-c1}
  Dado $x \in \bbZ, 0 \le x \le 2^n-1$, se define el complemento a 1 con $n$ bits de $x$, $C1_n(x)$, a la magnitud que se obtiene de sustituir todas las cifras a uno por cero y todas las cifras a cero por uno en la representación de $x$ en notación posicional en base 2 con $n$ cifras.
\end{definition}

El siguinte teorema permite calcular el complemento a 1 de un número de forma algebraica, sin necesidad de operar con números en base 2.

\begin{proposition}
  Dado $x \in \bbZ, 0 \le x \le 2^n-1$, se tiene que:
  \[
    C1_n(x) = 2^n - x - 1
  \]
  \begin{proof}
    Dado que $C1_n(x)$ se obtiene de complementar todas las cifras de $x$ escrito en base 2, es fácil ver que $x+C1_n(x)$ efectuada en base 2 tiene como resultado un número de $n$ cifras todas a uno, que es el mayor número representable en base 2 con $n$ cifras, por tanto:
    \begin{align*}
        x + C1_n(x) = 2^n - 1\\
        C1_n(x) = 2^n - x - 1
    \end{align*}
  \end{proof}
\end{proposition}

El siguiente corolario se obtiene directamente de la definición de complemento a 2 y proporciona un métod para el cálculo rápido del complemento a 2 en binario.

\begin{corollary}[Relación entre el complemento a 1 y el complemento a 2]\label{c1-c2-relation}
  Dado $x \in \bbZ, 0 \le x \le 2^n-1$, se tiene que:
  \begin{align*}
    C1_n(x) = C2_n(x) - 1\\
    C2_n(x) = C1_n(x) + 1
  \end{align*}
\end{corollary}

La siguiente proposición proporciona un método aun más rápido para obtener el complemento a 2 de un número.

\begin{proposition}
  Dado $x \in \bbZ, 0 \le x$ expresado en base 2, el complemento a 2 con $n$ bits de $x$ puede obtenerse con el siguiente procedimiento: comenzando por el bit menos significativo y avanzando hacia el más significativo, se copian todos los bits que sean cero hasta llegar al primer bit a uno. Se copia este primer bit a uno y a continuación se copian el resto de bits complementados.
  \begin{proof}
    La demostración de esta proposición se basa en el corolario~\ref{c1-c2-relation}. La operación $C1_n(x)+1$ transformará en 0 todos los bits a 1 de $C1_n(x)$ que se encuentren ajustados a la derecha y de forma consecutiva, mediante la propagación de acarreos consecutivos. Las sucesivas operaciones de acarreo se detienen al llegar al primer 0, que se transformará en 1 debido al acarreo. El resto de los bits de $C1_n(x)+1$ permancerán inalterados por la operación de suma al no existir acarreo que propagar. De esta forma, obtener el $C1_n(x)+1$ a partir del $C1_n(x)$ consiste en complementar todos los bits 1 de $C1_n(x)$ comenzando por la derecha (bit menos siginificativo) hasta llegar al primer 0, que también se complementa, dejando el resto de bits inalterados.

    Pero si tenemos en cuenta la definición~\ref{def-c1}, podemos llegar al mismo resultado para obtener el $C1_n(x)+1$ y, por tanto, el $C2_n(x)$ haciendo una transformación similar sobre los bits de $x$ teniendo en cuenta que cada bit de $x$ es el complemento del correspondiente bit de $C1_n(x)$. Esto es, en vez de complementar los bits a 1 empezando por la derecha hasta el primer 0 (inclusive), se dejan sin alterar los bits a 0 por la derecha hasta el primer 1 (inclusive), y en vez de dejar sin alterar el resto de bits, se complementan todos los bits restantes.
  \end{proof}
\end{proposition}

%%%%%%%%%%%%%%%%%%%%%%%%%%%%%%%%%%%%%%%%%%%%%%%%%%%%%%%%%%%%%%%%%%%%%%%%%%%%%%
\section{Extensión y reducción del número de bits en complemento a 2}
%%%%%%%%%%%%%%%%%%%%%%%%%%%%%%%%%%%%%%%%%%%%%%%%%%%%%%%%%%%%%%%%%%%%%%%%%%%%%%

Una operación frecuente en la represención binaria de números es tener que aumentar o disminuir el número de bits utilizados en la representación. Las siguientes propiedades proporcionan mecanismos para realizar estas operaciones directamente a partir de la representación en complemento a 2 del número.

El siguiente teorema nos dice que la representación en complemento a 2 de $x$ con $n+1$ bits se obtiene añadiendo un nuevo bit de signo igual al de la representación en complemento a 2 con $n$ bits de $x$.

\begin{theorem}[Extensión del signo en complemento a 2]
  Dado $x \in \bbZ$ representable en complemento a 2 con $n$ bits, $RC2_n(x)$ la reprsentanción en complemento a 2 de $x$ y $s$ el bit de signo de $x$, se cumple que:
  \[
      RC2_{n+1}(x) = s 2^n + RC2_n(x)
  \]
  \begin{proof}
      Distinguiremos dos casos:
    \begin{case}
      $0 \le x < 2^{n-1}$

      Dado que $0 \le x < 2^{n-1}$ se cumple también que $0 \le x < 2^n$ por lo que $x$ es representable complemento a 2 con $n+1$ bits. Por la definición~\ref{def-rep-ca2} se tiene que:
      \[
        RC2_{n+1}(x) = RC2_n(x) = x
      \]

      Y dado que $x \ge 0$, por el teorema~\ref{theorem-sign-bit} el bit de signo de la $RC2_n(x)$ es $s=0$, por tanto:
      \[
        RC2_{n+1}(x) = s 2^n + RC2_n(x)
      \]
    \end{case}
    \begin{case}
      $-2^{n-1} \le x < 0$

      Dado que $-2^{n-1} \le x < 0$ se cumple también que $-2^n \le x < 0$ por lo que $x$ es representable complemento a 2 con $n+1$ bits. Por la definición~\ref{def-rep-ca2} se tiene que:
      \[
        RC2_{n+1}(x) = 2^{n+1} + x = 2^n + 2^n + x\\
        RC2_n(x) = 2^n + x
      \]
      Luego:
      \[
        RC2_{n+1}(x) = 2^n + RC2_n(x)
      \]
      Y dado que $x < 0$, por el teorema~\ref{theorem-sign-bit} el bit de signo de la $RC2_n(x)$ es $s=1$, por tanto:
      \[
        RC2_{n+1}(x) = s 2^n + RC2_n(x)
      \]
    \end{case}
  \end{proof}
\end{theorem}

\begin{corollary}
  Dado $x \in \bbZ$ representable en complemento a 2 con n bits, $x$ será representable en complemento a 2 con $n-1$ bits si los dos bits más significativos de la representación en complemento a 2 de $x$ con $n$ bits escrita en base 2 son iguales, esto es, si el bit de signo no cambia al reducir en uno el número de bits de la representación.
\end{corollary}

%%%%%%%%%%%%%%%%%%%%%%%%%%%%%%%%%%%%%%%%%%%%%%%%%%%%%%%%%%%%%%%%%%%%%%%%%%%%%%
\section{Representación en complemento a 2 como código pesado}
%%%%%%%%%%%%%%%%%%%%%%%%%%%%%%%%%%%%%%%%%%%%%%%%%%%%%%%%%%%%%%%%%%%%%%%%%%%%%%

El siguiente resultado es de gran interés porque demuestra que la representación en complemento a 2 es un código pesado similar a la representación posicional en base 2. Este resultado puede servir como una definición alternativa de la representación en complemento a 2 cuando no se precisa o se desea entrar en los detalles de la relación de esta representación con las operaciones de complemento de números enteros.

El siguiente teorema también proporciona un mecanismo práctico para trabajar con representaciones en complemento a 2 (convertir un número a su representación y viceversa) si tener que pasar por las operaciones de complemento de números enteros.

\begin{theorem}[Representación en complemento a 2 como código pesado]
  Dado $x \in \bbZ$ representable en complemento a 2 con $n$ bits y $RC2_n(x)$ la magnitud de la reprsentanción en complemento a 2 de $x$ con $n$ bits formada por las cifras binarias $\{x_{n-1}, x_{n-2}, \ldots, x_1, x_0\}$, se tiene que:
  \[
      x = x_{n-1} (-2^{n-1})+x_{n-2} 2^{n-2}+ \cdots + x_1 2^1+x_0
  \]
  \begin{proof}
    Distinguimos los casos en que $x$ sea positivo (o cero) y $x$ negativo.

    \begin{case}
      $0 \le x < 2^{n-1}$

      Según la definición~\ref{def-rep-ca2} se tiene que $RC2_n(x) = x$. Por el teorema~\ref{theorem-sign-bit}, $x_{n-1}$ es el bit de signo de $x$ y es cero, por tanto se cumple que:
      \begin{align*}
        x = x_{n-2} 2^{n-2}+ \cdots + x_1 2^1+x_0\\
        x = x_{n-1} (-2^{n-1})+x_{n-2} 2^{n-2}+ \cdots + x_1 2^1+x_0
      \end{align*}
    \end{case}
    \begin{case}
      $-2^{n-1} \le x < 0$

      Según la definición~\ref{def-rep-ca2} se tiene que $RC2_n(x) = 2^n + x$. Por el teorema~\ref{theorem-sign-bit}, $x_{n-1}$ es el bit de signo de $x$ y es uno. Si la $RC2_n(x)$ es:
      \[
        RC2_n(x) = 2^n + x = 2^{n-1}+x_{n-2}2^{n-2}+\cdots+x_1 2^1+x_0
      \]
      Despejando $x$:
      \[
        x = 2^{n-1}-2^n + x_{n-2} 2^{n-2}+ \cdots +x_1 2^1+x_0
      \]
      Teniendo en cuenta que $2^{n-1}-2^n = -2^{n-1}$ y que $x_{n-1}=1$, se tiene que:
      \[
        x = x_{n-1} (-2^{n-1})+x_{n-2} 2^{n-2}+ \cdots + x_1 2^1+x_0
      \]
    \end{case}
  \end{proof}
\end{theorem}
\end{document}
